% \chapter{ABSTRACT}
Historically, low-lying excitations in deformed nuclei have presented unique challenges in the context of nuclear structure. Early formulations of a geometric model surmise that quadrupole and octupole surface shape vibrations can be superimposed on top of a pre-existing prolately deformed nucleus. As the experimental evidence trickles in, this idea of strongly collective behavior becomes obscured for a multitude of reasons. Successful interpretation of the feasability of quadrupole vibrations hinges on the extraction of transition probabilities between nuclear states, which can be determined via the measurement of level lifetimes. Femtosecond to picosecond range nuclear lifetimes can be measured with the Doppler Shift Attenuation Method via Inelastic Neutron Scattering at the University of Kentucky Accelerator Laboratory, where we have performed a series of measurements of nuclear lifetimes in two rare-earth nuclei, $^{160}$Gd and $^{162}$Dy. This work discusses the lifetime measurements made, analysis techniques implemented, and nuclear structure as it pertains to single- and multi-phonon configurations of the previously mentioned deformed nuclei.


% The focus of this work deals with the nature of low-lying excitations in well-deformed nuclei. Historically, these states have been a veritable enigma in understanding the specific structure effects that manifest in the nucleus. The dynamic oscillations of various order (quadrupole, octupole, etc) BLAH BLAH BLAH